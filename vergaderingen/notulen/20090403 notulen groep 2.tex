%
%  untitled
%
%  Created by Etienne van Delden on 2009-04-02.
%  Copyright (c) 2009 __MyCompanyName__. All rights reserved.
%
\documentclass[]{article}

% Use utf-8 encoding for foreign characters
\usepackage[utf8]{inputenc}

% Setup for fullpage use
%\usepackage{fullpage}

% Uncomment some of the following if you use the features
%
% Running Headers and footers
%\usepackage{fancyhdr}
\usepackage{hyperref}

% Multipart figures
%\usepackage{subfigure}

% More symbols
\usepackage{amsmath}
\usepackage{amssymb}
%\usepackage{latexsym}

% Surround parts of graphics with box
\usepackage{boxedminipage}

% Package for including code in the document
\usepackage{listings}

% If you want to generate a toc for each chapter (use with book)
%\usepackage{minitoc}

% This is now the recommended way for checking for PDFLaTeX:
\usepackage{ifpdf}

%\newif\ifpdf
%\ifx\pdfoutput\undefined
%\pdffalse % we are not running PDFLaTeX
%\else
%\pdfoutput=1 % we are running PDFLaTeX
%\pdftrue
%\fi

\ifpdf
\usepackage[pdftex]{graphicx}
\else
\usepackage{graphicx}
\fi
\title{Notulen vergadering Groep 2 (2IO45)}
\author{ Etienne van Delden }

\date{2009-04-02}

\begin{document}

\ifpdf
\DeclareGraphicsExtensions{.pdf, .jpg, .tif}
\else
\DeclareGraphicsExtensions{.eps, .jpg}
\fi

\maketitle

\section{Opening vergadering} % (fold)
\label{sec:opening_vergadering}
\begin{tabular}{ll}
  Datum:      & 03-04-2009\\
  Begin tijd: & 14:02\\
  Voorzitter: & Anson van Rooij \\
  Notulist:   & Etienne van Delden\\
  \multicolumn{2}{l}{Aanwezigheid leden: Iedereen is aanwezig}
\end{tabular}

% section opening_vergadering (end)

\section{Opmerkingen vorige notulen} % (fold)
\label{sec:opmerkingen_vorige_notulen}
  De notulen zijn goedgekeurd, er was alleen een typo.
% section opmerkingen_vorige_notulen (end)

\section{Geplande Mededelingen / Vragen} % (fold)
\label{sec:geplande_mededelingen}
  \begin{itemize}
    \item Er wordt gevraagd of wij pyGame en de OpenDynamics Engine mogen gebruiken. De tutor zal er naar kijken, maar waarschijnlijk kunnen we ze gewoon gebruiken.
  \end{itemize}
% section geplande_mededelingen (end)

\section{Overige Mededelingen} % (fold)
\label{sec:overige_mededelingen}

\subsection*{Tutor} % (fold)
\label{sub:tutor}
\begin{itemize}
  \item Er is aanstaande dinsdag het studentenoverleg, de locatie is nog niet bekend. Etienne zal deze bijwonen.
  \item Het werkplan is goedgekeurd.
  \item Vandaag is de deadline voor de specificatie.
\end{itemize}
% subsection tutor (end)
% section overige_mededelingen (end)

\section{Planning komende week} % (fold)
\label{sec:planning_komende_week}
  \begin{itemize}
    \item Jeroen kijkt naar een nieuw design voor het netwerk; een inter-connected netwerk, zodat het makkelijker is om verdwenen tokens te signaleren en om het spel te joinen. Het verliezen van tokens kan echter aan DirectPlay liggen.

  \end{itemize}
% section planning_komende_week (end)



\section{W.V.T.T.K} % (fold)
\label{sec:w_v_t_t_k}
  \begin{itemize}
    \item Jeroen kent iemand die 3d modellen voor ons wil maken in 3DS Max. Er is code om die modellen in OpenGL in te lezen.
    \item 8 Mei is de deadline voor het ontwerp document.
    \item Volgende week is het Goede vrijdag en dus geen les. Er zal dan ook geen vergadering zijn.
  \end{itemize}
% section w_v_t_t_k (end)

\section{Sluiting} % (fold)
\label{sec:sluiting}
\emph{om 14:14}
% section sluiting (end)

\end{document}
