\documentclass[]{article}

\usepackage[english]{babel}
%\usepackage{fullpage}
\usepackage[utf8]{inputenc}
\usepackage{a4wide,graphicx,fancyhdr,clrscode,amsmath,amssymb,tabularx}
\usepackage{listings}
\usepackage{color}

\title{Shape of messages from input module and network module to engine}

\begin{document}

\section{input messages}

This section concerns Ontwerpdocument, 4.2 game related messages, input.\\
\\
Input-messages are lists which have three elements - the first is a Key:

\subsection{the Key}

In case the Key represents a mouse click, it is a list which has three elements, the first of which is the string `ml'or 'mr', the second of which is an integer; the change in screen coordinate X, and the third of which is also an integer; the change in screen coordinate Y.\\
\\ 
In case the Key represents a chat message, it is a list which has two elements, the first of which is the string `c', the second of which is a the string to be sent.\\
\\
In case the Key represents something else, it is a list which has two elements, the first of which is the string `b', the second of which is a string as follows:

\begin{itemize}
    \item Accelerate Forward: `AcceForwZ'
    \item Accelerate Backward: `AcceBackZ'
    \item Step Sideways Left: `StepLeftZ'
    \item Step Sideways Right: `StepRghtZ'
    \item Drop Bomb: `DropBombZ'
    \item Activate Powerup 1: `Powerup1Z'
    \item Activate Powerup 2: `Powerup2Z'
    \item Activate Powerup 3: `Powerup3Z'
    \item Minimap Toggle: `MinimapTZ'
\end{itemize}


Where Z is On for key pressed or Off for key released.

\subsection{the Player ID}

The second part of an Input-message is a Player ID, which is an integer; the ID of the sending player.

\subsection{the Order}

The third part if an Input-message is an Order, which is an integer; the place of the input message in relation to other input messages sent at the same game time.

\end{document}
